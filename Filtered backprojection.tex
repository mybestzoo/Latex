\documentclass[12pt]{iopart}
\bibliographystyle{unsrt} 

\usepackage{amssymb,amsfonts}
 \expandafter\let\csname equation*\endcsname\relax
  \expandafter\let\csname endequation*\endcsname\relax
\usepackage{amsmath}
%\usepackage{iopams}
\usepackage{graphicx,float}
\usepackage[caption = false]{subfig}
\usepackage{subcaption}
%\usepackage{setspace}
%\usepackage{cite}
%\usepackage{indentfirst}
\usepackage{color,soul}

\newtheorem{theorem}{Theorem}
\newtheorem{lemma}{Lemma}
\newtheorem{conseq}{Consequence}
\newenvironment{proof}
{\par\noindent{\bf Proof}}
{\hfill$\scriptstyle\blacksquare$}

\begin{document}

\begin{lemma} (Theorem 3.97 \cite{MA})
\label{isometry}
If $f\in L_2(\mathbb R^d)$ and $\Lambda^{k/2}Pf\in L_2(\mathbb TG_{k,d})$, then
$$\|\Lambda^{k/2}Pf\|^2_{L_2(TG_{k,d})}=(2\pi)^k\gamma_{d-k,d}\|f\|^2_{L_2(\mathbb R^d)}.$$
\end{lemma}

\begin{proof}
\begin{multline*}
\|\Lambda^{k/2}Pf\|^2_{L_2(TG_{k,d})}=\int_{G_{k,d}}\int_{\pi^\perp}\left|\Lambda^{k/2}Pf(\pi,x'')\right|^2dx''d\pi=\\
=\int_{G_{k,d}}\int_{\pi^\perp}\left|\widehat{\Lambda^{k/2}Pf}(\pi,\xi'')\right|^2d\xi''d\pi=\int_{G_{k,d}}\int_{\pi^\perp}(2\pi)^k|\xi''|^k\left|\widehat{f}(\xi'')\right|^2d\xi''d\pi=\\
=(2\pi)^k\gamma_{d-k,d}\int_{\mathbb R^d}\left|\widehat{f}(\xi)\right|^2d\xi =(2\pi)^k\gamma_{d-k,d}\|f\|^2_{L_2(\mathbb R^d)},$$
\end{multline*}
where we used Plancherel's theorem, theorem \ref{projection} and equality \eqref{integral}.
\end{proof}

\begin{lemma}
\label{lemma2}
If $P^\#\Lambda^kg\in L_2(\mathbb R^d)$ and $\Lambda^{k/2}g\in L_2(TG_{k,d})$, then
$$\|P^\#\Lambda^kg\|_{L_2(\mathbb R^d)}\le\sqrt{(2\pi)^k\gamma_{d-k,d}}\|\Lambda^{k/2}g\|_{L_2(\mathbb TG_{k,d})}.$$
\end{lemma}

\begin{proof}
\begin{multline*}
\|P^\#\Lambda^kg\|_{L_2(\mathbb R^d)}^2=|<P^\#\Lambda^kg,P^\#\Lambda^kg>_{L_2(\mathbb R^d)}|=|<PP^\#\Lambda^kg,\Lambda^kg>_{L_2(TG_{k,d})}|=\\
=|<\Lambda^{k/2}PP^\#\Lambda^kg,\Lambda^{k/2}g>_{L_2(TG_{k,d})}|\le\|\Lambda^{k/2}PP^\#\Lambda^kg\|_{L_2(TG_{k,d})}\|\Lambda^{k/2}g\|_{L_2(TG_{k,d})}=\\
=\sqrt{(2\pi)^k\gamma_{d-k,d}}\|P^\#\Lambda^kg\|_{L_2(\mathbb R^d)}\|\Lambda^{k/2}g\|_{L_2(\mathbb TG_{k,d})},
\end{multline*}
where we used lemma \ref{isometry} and duality between $P$ and $P^\#$ \ref{duality}.
\end{proof}

Define method by 

$$m_ag(x) = \frac{1}{(2\pi)^k\gamma_{d-k,d}}P^\#\Lambda^k[a g](x).$$

Note, that $f(x)=\frac{1}{(2\pi)^k\gamma_{d-k,d}}P^\#\Lambda^kPf(x)$.
By lemma \ref{lemma2} we have
\begin{multline*}
    \|f-m_ag\|^2=\|\frac{1}{(2\pi)^k\gamma_{d-k,d}}P^\#\Lambda^k(Pf-ag)\|^2\le\frac{1}{(2\pi)^k\gamma_{d-k,d}}\|\Lambda^{k/2}(Pf-ag)\|^2=\\
    =\frac{1}{(2\pi)^k\gamma_{d-k,d}}\int_{G_{k,d}}\int_{\pi^\perp}|\xi''|^k\left|\widehat{Pf_\pi}(\xi'')-a(\xi'')\widehat g_\pi(\xi'')\right|^2d\xi'' d\pi=\\
    =\frac{1}{\gamma_{d-k,d}}\int_{G_{k,d}}\int_{\pi^\perp}|\xi''|^k\left|\widehat{f}(\xi'')-(2\pi)^{-k/2}a(\xi'')\widehat g_\pi(\xi'')\right|^2d\xi'' d\pi.
\end{multline*}







From the general theory of Helgason \cite{H} we derive the following formula for the integration over $\mathbb R^d$ (corollary 2.4 from \cite{K})
\begin{equation}
\label{integral}
\int_{\mathbb R^d}f(x)dx=\frac{1}{\gamma_{d-k,d}}\int_{G_{k,d}}\int_{\pi^\perp}|x''|^kf(x'')dx''d\pi,
\end{equation}
where
$$
\gamma_{k,d}=|G_{k-1,d-1}|.
$$

One important relation between the k-plane transform and the Fourier transform
$$\quad \widehat f(\xi)=(2\pi)^{-d/2}\int_{\mathbb R^d}e^{-ix\xi}f(x)dx$$
is  known as the projection-slice theorem. 

\begin{theorem}
\label{projection}
If $f\in L_1(\mathbb R^d)$, then
$$\widehat{(P_\pi f)}(\xi'')=(2\pi)^{k/2}\widehat f(\xi''),\quad \xi''\in\pi^\perp.$$
\end{theorem}

\begin{proof}
Now we show, that the error of the methods \eqref{method} is equal to the achieved estimate.
We have
\begin{multline*}
  \|(-\Delta)^{\beta/2}f-m_a(g)\|^2_{L_2(\mathbb R^d)}=\|\widehat{(-\Delta)^{\beta/2}f}-\widehat{m_a(g)}\|^2_{L_2(\mathbb R^d)}=\\
  =\int_{G_{k,d}}\int_{\pi^\perp}\frac{|\xi''|^k}{\gamma_{d-k,d}}\left||\xi''|^\beta \widehat f(\xi'')-(2\pi)^{-k/2}a(\xi'')\widehat{g_\pi}(\xi'')\right|^2d\xi'' d\pi =\\
  =\int_{G_{k,d}}\int_{\pi^\perp}\frac{|\xi''|^k}{\gamma_{d-k,d}}\left|a(\xi'')(2\pi)^{-k/2}\left(\widehat{g_\pi }(\xi'')-(2\pi)^{k/2}\widehat 
      f(\xi'' )\right)+\widehat
    f(\xi'')\left(a(\xi'')-|\xi''|^\beta\right)\right|^2d\xi'' d\pi .
\end{multline*}
Transform this expression using the Cauchy-Schwarz inequality $|qz|\leqslant |z||q|$ applied to vectors
\[
z=\left((2\pi)^{-k/2}\frac{a(\xi'')}{\sqrt{\widehat\lambda_2}},\frac{\sqrt{\gamma_{d-k,d}}}{|\xi''|^{\frac{k+2\alpha}{2}}}\frac{(a(\xi'')-|\xi''|^\beta)}{\sqrt{\widehat\lambda_1}}\right),
\]
\[
q=\left(\left(\widehat{g_\pi }(\xi'')-(2\pi)^{k/2}\widehat
    f(\xi'' )\right)\sqrt{\widehat\lambda_2},\frac{|\xi''|^{\frac{k+2\alpha}{2}}}{\sqrt{\gamma_{d-k,d}}}\sqrt{\widehat\lambda_1}\widehat
  f(\xi'' )\right).
\]
We obtain
\begin{multline*}  
  \|(-\Delta)^{\beta/2}f-m_a(g)\|^2_{L_2(\mathbb R^d)}\leqslant  \\
  \leqslant \int_{G_{k,d}}\int_{\pi^\perp}
  A(\xi'')\left(\frac{|\xi''|^{k+2\alpha}}{\gamma_{d-k,d}}\widehat\lambda_1|\widehat
    f(\xi'')|^2+\left|\widehat{g_\pi
      }(\xi'')-(2\pi)^{k/2}\widehat f(\xi''
      )\right|^2\widehat\lambda_2\right)d\xi'' d\pi ,
\end{multline*}
where
  \[
  A(\xi'')=\frac{|\xi''|^k}{\gamma_{d-k,d}}\left((2\pi)^{-k}\frac{a^2(\xi'')}{\widehat\lambda_2}+\frac{\gamma_{d-k,d}}{|\xi''|^{k+2\alpha}}\frac{(a(\xi'')-|\xi''|^\beta)^2}{\widehat\lambda_1}\right).
  \]
  The condition \eqref{a} is equivalent to $A(\xi'')\leqslant 1$, which leads to $ \|(-\Delta)^{\beta/2}f-m_a(g)\|^2_{L_2(\mathbb R^d)}\leqslant
  \widehat\lambda_1+\widehat\lambda_2\delta^2.$
Thus, we end with the proof.

\end{proof}

The design of the optimal methods actually adds a filter $a(\xi'')$ to the projection theorem and instead of the $k$-plane transform we deal with its Fourier image. This filter defines the amount of information that we use for the optimal recovery. When $a(\xi'')$ can be chosen equal to $|\xi''|^\beta$, the optimal method turns into the exact formula $\widehat{m_a(g)}(\xi'')=(2\pi)^{-k/2}|\xi''|^\beta\widehat{g_\pi }(\xi'')$ , which means that the corresponding volume of information doesn't need to be filtered. On the other hand some information is unnecessary as it may not be used by the optimal method, when $a(\xi'')$ can be equal to $0$. The following consequence shows that for sufficiently small  $|\xi''|$ information $\widehat{g_\pi}(\xi'')$ doesn't need to be filtered and, on the contrary, for large  $|\xi''|$ the information is useless, as it  has no effect on the error of the optimal recovery.

%CONS1
\begin{conseq}
\label{cons}
In the conditions of the Theorem \ref{theorem} the following methods are optimal $$
\widehat{m_a(g)}(\xi'')=(2\pi)^{-k/2}a(\xi'')\widehat{g_\pi }(\xi''),\quad \xi''\in\pi^\perp,$$ where
  \[
a(\xi'')=
  \begin{cases}
    1& ,|\xi''|\le \tau_1,\\
    \frac{\widehat\lambda_2|\xi''|^\beta}{\widehat\lambda_1t(|\xi''|)+\widehat\lambda_2}+\varepsilon(\xi'')\frac{\sqrt{\widehat\lambda_1\widehat\lambda_2}|\xi''|^\alpha}{\widehat\lambda_1t(|\xi''|)+\widehat\lambda_2}\sqrt{t(|\xi''|)\widehat\lambda_1+\widehat\lambda_2-y(|\xi''|)}& ,\tau_1 \le|\xi''|\le\tau_2,\\
    0 &,|\xi''|\ge\tau_2,
  \end{cases}
\]
$\varepsilon$ is an arbitrary function satisfying $\|\varepsilon\|_{L_\infty(\mathbb R^d)}\le 1$, $\tau_1=((2\pi)^k\widehat\lambda_2\gamma_{d-k,d})^\frac{1}{k+2\beta}$, $\tau_2=\widehat\lambda_1^{\frac{-1}{2(\alpha-\beta)}}.$
\end{conseq}

\begin{proof}
As we've seen in the proof of the Theorem \ref{theorem} the condition on $a(\xi'')$ for the method $m_a(g)$ to be optimal is $A(\xi'')\leqslant 1$. Put $a(\xi'')=|\xi''|^\beta$ to this inequality and solve it for $\xi''$ to obtain $|\xi''|\le ((2\pi)^k\widehat\lambda_2\gamma_{d-k,d})^\frac{1}{k+2\beta}$. By the analogue put $a(\xi'')=0$,
  then $A(\xi'')\leqslant 1$ is true when $|\xi''|\geqslant
  \widehat\lambda_1^{\frac{-1}{2(\alpha-\beta)}}$.
\end{proof}

\begin{figure}[h]
\centering
\includegraphics[scale=0.4]{cons1.png}
\caption{The optimal values of filter $a$ lie between two graphs: one represents function $a$ from theorem \ref{theorem} for $\epsilon(\xi'')=1$, another for $\epsilon(\xi'')=-1$. Here $d=2, k=1, \beta=0, \alpha=2, \delta=1$.}
\label{pic1}
\end{figure}

An obvious observarion here is that the methods from the Consequence \ref{cons} give the result of the optimal recovery as a bandlimited function. Another application of the Theorem \ref{theorem} is a new inequality for the norms of the k-plane transform and the degree of the Laplace operator.


%CONS2
\begin{conseq}
\label{cons2}
The following exact inequality takes place for a function $|\xi|^\beta\widehat f(\xi)\in L_2(\mathbb R^d)$, $|\xi|^\alpha\widehat f(\xi)\in L_2(\mathbb R^d)$, $Pf\in L_2(TG_{k,d})$, $0\le\beta<\alpha$:
\[
\|(-\Delta)^{\beta/2}f\|_{L_2(\mathbb R^d)}\leqslant
((2\pi)^k\gamma_{d-k,d})^{\frac{\beta-\alpha}{2\alpha+k}}\|Pf\|_{L_2(TG_{k,d})}^{\frac{2(\alpha-\beta)}{2\alpha+k}}\|(-\Delta)^{\alpha/2}f\|_{L_2(\mathbb
  R^d)}^\frac{k-2\beta}{2\alpha+k}.
\]
\end{conseq}

\begin{proof}
From the solution of the dual problem in Theorem \ref{theorem} it follows, that \linebreak
 $\|(-\Delta)^{\beta/2}u\|_{L_2(\mathbb R^d)}\leqslant E(\delta)=
  ((2\pi)^k\gamma_{d-k,d})^{\frac{\beta-\alpha}{2\alpha+k}}\delta^{\frac{2(\alpha-\beta)}{2\alpha+k}}$, 
  when the following constraints are satisfied: $\|Pu\|_{L_2(TG_{k,d})}=\delta$ and
  $\|(-\Delta)^{\alpha/2}u\|_{L_2(\mathbb R^d)}=1$. So the expression can be presented as \linebreak
$\|(-\Delta)^{\beta/2}u\|_{L_2(\mathbb R^d)}\leqslant
  ((2\pi)^k\gamma_{d-k,d})^{\frac{\beta-\alpha}{2\alpha+k}}\|Pu\|_{L_2(TG_{k,d})}^{\frac{2(\alpha-\beta)}{2\alpha+k}}$.
 Now we put
 $u(x)=\frac{f(x)}{\|(-\Delta)^{\alpha/2}f\|_{L_2(\mathbb R^d)}}$, $f\ne 0$ to obtain
\[
\|(-\Delta)^{\beta/2}f\|_{L_2(\mathbb R^d)}\leqslant
((2\pi)^k\gamma_{d-k,d})^{\frac{\beta-\alpha}{2\alpha+k}}\|Pf\|_{L_2(TG_{k,d})}^{\frac{2(\alpha-\beta)}{2\alpha+k}}\|(-\Delta)^{\alpha/2}f\|_{L_2(\mathbb
  R^d)}^\frac{k-2\beta}{2\alpha+k}.
\]
\end{proof}

As we already mentioned some particular cases of the presented problem bring the most interest. In computerized tomography theory the general problem is to recover a function itself from different sort of tomographic data, which corresponds to $\beta=0$ in our notations. A special case of $\beta=1$ (recovery of the $\sqrt{-\Delta}$ operator value) is studied in the local tomography theory. The results for the X-Ray transform are totally correspond to the theorem \ref{theorem}, consequences \ref{cons} and \ref{cons2} by putting $\beta=0$ and $k=1$. The case of the Radon transform needs an additional remark as its usual definition differs from the one that we use here. Let $(s,\theta)\in\mathbb R\times\mathbb S^{d-1}=Z$ and $x\in\mathbb R^d$ then the Radon transform is defined by the formula 
$$Rf(\theta,s)=\int_{x\theta=s}f(x)dx.$$
Clearly the function $Rf(\theta,s)$ has the same value as $Pf(\pi,x'')$, where $\pi=\{x\in\mathbb R^d | x\theta=0\}$, $x''=s\theta$ and also
\begin{equation}
\label{norms}
\|Rf\|_{L_2(Z)}=\sqrt{2}\|Pf\|_{L_2(TG_{k,d})},\quad k=d-1.
\end{equation}
Which means, that class $W$ for $\beta=0$ can be equivalently presented as 
$$ W=\{f\in L_2(\mathbb R^d) :
\|(-\Delta)^{\alpha/2}f\|_{L_2(\mathbb R^d)}\leqslant  1;\quad Rf\in L_2(Z) \},\quad\alpha>0$$
and the error of the optimal recovery is
$$
E(\delta)=\inf_{m:L_2(Z)\rightarrow L_2(\mathbb R^d)}\sup_{
  \begin{smallmatrix}
f\in W, g\in L_2(Z)\\ 
\|Rf-g\|_{L_2(Z)}\leqslant \delta
\end{smallmatrix}} ||f-m(g)||_{L_2(\mathbb R^d)}.
$$
From \eqref{norms} it follows that the solution of this problem is equivalent to the solution in theorem \ref{theorem} where $\delta$ is substituted by $\delta/\sqrt{2}$ in the expressions for $\lambda_1$, $\lambda_2$ and $E(\delta)$
$$
 \widehat\lambda_1=(2\pi)^{\frac{2\alpha(1-d)}{2\alpha+d-1}}\frac{d-1}{2\alpha+d-1}\left(\frac{\delta}{\sqrt{2}}\right)^\frac{4\alpha}{2\alpha+d-1},\quad \widehat\lambda_2=(2\pi)^{\frac{2\alpha(1-d)}{2\alpha+d-1}}\frac{2\alpha}{2\alpha+d-1}\left(\frac{\delta}{\sqrt{2}}\right)^\frac{2(1-d)}{2\alpha+d-1}, 
$$
$$
E(\delta)=\sqrt{\widehat\lambda_1+\widehat\lambda_2\frac{\delta^2}{2}}=(2\pi)^{\frac{\alpha(1-d)}{2\alpha+d-1}}\left(\frac{\delta}{\sqrt{2}}\right)^{\frac{2\alpha}{2\alpha+d-1}}.
$$
The optimal methods will take the form 
$$
  \widehat{m_a(g)}(\sigma\theta)=(2\pi)^{(1-d)/2}a(\sigma)\widehat{g_\theta }(\sigma),\quad \sigma\in[0,\infty],\quad \theta\in\mathbb S^{d-1},
$$
where $g_{\theta}(s)=g(\theta,s)$ and function $a$ can be presented according to the consequence \ref{cons} as
$$
a(\sigma)=
  \begin{cases}
    1& ,\sigma\le (2\pi)\widehat\lambda_2^\frac{1}{d-1},\\
    \frac{\widehat\lambda_2}{\widehat\lambda_1t(\sigma)+\widehat\lambda_2}+\varepsilon(\sigma)\frac{\sqrt{\widehat\lambda_1\widehat\lambda_2}\sigma^\alpha}{\widehat\lambda_1t(\sigma)+\widehat\lambda_2}\sqrt{t(\sigma)\widehat\lambda_1+\widehat\lambda_2-y(\sigma)}& ,(2\pi)\widehat\lambda_2^\frac{1}{d-1} \le\sigma\le\widehat\lambda_1^{\frac{-1}{2\alpha}},\\
    0 &,\sigma\ge\widehat\lambda_1^{\frac{-1}{2\alpha}},
  \end{cases}
$$
$\varepsilon$ is an arbitrary function satisfying $\|\varepsilon\|_{L_\infty(\mathbb R)}\le 1$.

Finally, the Radon transform satisfies the inequality
$$
\|f\|_{L_2(\mathbb R^d)}\leqslant
(2\pi)^{\frac{\alpha(1-d)}{2\alpha+d-1}}2^{\frac{-\alpha}{2\alpha+d-1}}\|Rf\|_{L_2(Z)}^{\frac{2\alpha}{2\alpha+d-1}}\|(-\Delta)^{\alpha/2}f\|_{L_2(\mathbb
  R^d)}^\frac{d-1}{2\alpha+d-1},\quad \alpha>0.
$$




\section*{References}

\bibliography{document}

\end{document}
