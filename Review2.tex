\documentclass{article}
\usepackage{color}
\usepackage{amssymb,amsfonts}

\begin{document}

\begin{center}
{\Large Response to the referee's report 2}
\end{center}
\vspace{5mm}

I'd like to thank referee for a careful attitude to the manuscript and especially the last remark, which led to a major improvement in the paper. Hereby I present my feedback on the remarks in the review (numeration preserved):

\begin{enumerate}
\item Notation $TG_{k,d}$ changed to $\mathcal G_{k,d}$;
\item "Consequence 1"/"Consequence 2" changed to "Corollary 1"/"Corollary 2";
\item One of the methods in local tomography theory is the so-called Lambda tomography. Instead of function $f$ itself it deals with the related function $Lf = \Lambda f+\mu\Lambda^{-1}f$, where $\Lambda = \sqrt{-\Delta}$. This has the advantege that the reconstruction is strictly local in the sense that computation of $Lf(x)$ requires only integrals over lines passing arbitrarily close to $x$. The details can be found in paper:
A. Faridani, F. Keinert, F. Natterer, E.L. Ritman, K.T. Smith, \textit{Local and global tomography}, in \textit{Signal processing II}, Springer-Verlag, New York, 1990.
Corresponding explanations added to the text.

\item To clarify the notations we specify that when applied to $g(\pi,x'')$ the Fourier transform and operator $\Lambda^k=(-\Delta)^{k/2}$ act on the second variable and we denote $g_\pi(x'')=g(\pi,x'')$. Thus $\widehat g_\pi(\xi'')$ is the Fourier transform of $g_\pi(x'')$. The idea of the optimal methods is to use the projection-slice theorem, apply filter $a(\xi'')$ to $\widehat g_\pi(\xi'')$ and then perform the inverse Fourier transform. However, as it was mentioned in the review, the direct application of the inverse Fourier transform is not correct, as $\widehat g_\pi(\xi'')$ is defined for $\xi''\in\pi^\perp$ and therefore doesn't specify a unique function on $\mathbb R^d$ (an exception is the Radon transform, for which $k=d-1$ and for every point in $\mathbb R^d$ there is a unique line $\pi^\perp$ passing through it). To avoid the direct application of the inverse Fourier transform we use the back-projection operator $P^\#$ to present the optimal methods in a form
    $$
	m_a(g)(x) = \frac{1}{(2\pi)^k\gamma_{d-k,d}}[P^\#\Lambda^ku](x),
	$$
where	
	$$\widehat{u}(\xi'')=|\xi''|^\beta a(\xi'')\widehat{g_\pi }(\xi''),\quad \xi''\in\pi^\perp.$$
The conditions on filter $a(\xi'')$ from theorem 2 guarantee that this method (series of methods) presents an optimal approximation of the considered set $W$ in $L_2$ norm.	Corresponding changes added to the statement and the proof of theorem 2. 
\end{enumerate}

\end{document}
