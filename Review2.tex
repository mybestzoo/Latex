\documentclass{article}
\usepackage{color}

\title{Response to the referee's report 2}
\date{}

\begin{document}
\maketitle

I'd like to thank referee for a careful attitude to the manuscript and present my feedback on the remarks in the review (numeration preserved):

\begin{enumerate}
\item Notation
\item Corollary
\item One of the methods in local tomography theory is the so-called Lambda tomography. Instead of function $f$ itself it deals with the related function $Lf = \Lambda f+\mu\Lambda^{-1}f$, where $\Lambda = \sqrt{-\Delta}$. This has the advantege that the reconstruction is strictly local in the sense that computation of $Lf(x)$ requires only integrals over lines passing arbitrarily close to $x$. The details can be found in paper:

A. Faridani, F. Keinert, F. Natterer, E.L. Ritman, K.T. Smith, \textit{Local and global tomography}, in \textit{Signal processing II}, Springer-Verlag, New York, 1990 

{\color{red}{Check the publication}}
  
\item Fourier transform
\end{enumerate}

\end{document}
