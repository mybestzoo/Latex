\documentclass{article}
\usepackage{color}
\usepackage{amssymb,amsfonts}

\begin{document}

\begin{center}
{\Large Response to the referee's report 1}
\end{center}
\vspace{5mm}

I'd like to thank referee for a careful attitude to the manuscript and present my feedback on the remarks in the review:

\begin{enumerate}
    \item We use an approach to optimal recovery problems based on the general extremal problems theory which was developed in works of K.Yu. Osipneko and G.G. Magatil-Il'yaev. In this framework the Lagrange principal is exploited in order to solve the optimal recovery problems. However this is a general principle, but not a mathematical result which allows to achieve explicit solutions for the problems of optimal recovery with uncertain information. Thereby any particular result in this field is important in itself. Every optimal recovery problem starts with a certain class of functions, which defines both the error of the optimal recovery and the optimal methods. Though in papers [10,11] and current paper we consider a similar information operators (but not the same), the classes of functions are different and it brings a significant difference in the nature of optimal methods and the techniques used to solve the corresponding optimal recovery problem. In paper [10] we consider the Hardy space $h_2$ of harmonic functions in a unit ball in $\mathbb R^d$, while the information is the radial integration operator, measured with an error. Paper [11] considers the same class of functions, but under the action of the Radon transform. The optimal methods in these two problems arise from the structure of the Hardy space and decomposition of functions into special series by spherical harmonics. Current paper deals with a different class of functions, which is free from the conditions of the Hardy space, that are not typical to computerized tomography problems. We also consider the k-plane transform that include the Radon transform as a particular case $k=d-1$. However that doesn't mean that current paper generalizes results of previous works [10,11], as both the optimal errors and the optimal methods in those papers are different and results don't follow from each other. Paper "On a problem of optimal recovery and Kolmogorov type inequalities on an interval" deals with optimal recovery problem for the k-th derivative of the function on an interval from the inaccurate information on the function itself. It intersects with the current paper only in the general construction of the problem statement and the formulation of the Lagrange principal that we also use.
    
    All results presented in the current paper are new and independent from the results of other published papers. They consist of the explicit form of the optimal recovery error, new optimal methods for function recovery from the k-plane transform measured with an error and a new inequality for the k-plane transform.
    
    From the point of view of applications we consider a more "natural" class of functions and a more general information operator than in previous papers. However mathematically these are different problems and none of them is a consequence of of another.
    
    
    
    
    
    
    
    
    
    
    
    
    \item Signs $\le$/$\ge$ changed to $\leqslant$/$\geqslant$; "Consequence 1"/"Consequence 2" changed to "Corollary 1"/"Corollary 2"; "relation" substituted by "equation"; other misprints are fixed. 
\end{enumerate}

\end{document}
