\documentclass{article}
\usepackage[T1,T2A]{fontenc}
\usepackage[utf8]{inputenc}
\usepackage{amssymb,amsfonts}
\usepackage{amsmath}
\usepackage{color,soul}
\usepackage[russian,english]{babel}

\begin{document}
В основной теореме оптимальные методы задаются формулой 
\begin{equation}
\label{method}
\hat{m_a(g)}(\xi'')=(2\pi)^{-k/2}a(\xi'')\hat{g_\pi}(\xi''),\quad \xi''\in\pi^\perp.
\end{equation}
Здесь $g\in L_2(TG_{k,d})$, $TG_{k,d}=\{(\pi,x''):\pi\in G_{k,d}, x''\in\pi^\perp\}$, $G_{k,d}$ - многообразие Грассмана всех k-мерных подространств в $\mathbb R^d$, мера которого конечна $|G_{k,d}|<\infty$. Функция $\hat{g_\pi}(\xi'')$ является преобразованием Фурье функции $g(\pi,x)$, взятом по второй переменной. Проблема в том, что равенство \eqref{method}, возможно, не определяет функцию $m_a(g)$ как элемент $L_2(\mathbb R^d)$. Можно показать, что правая часть равенства принадлежит $L_2(\pi^\perp)$ для почти всех $\pi\in G_{k,d}$:

\begin{itemize}
    \item Вместо функции, заданной на $TG_{k,d}$ можно рассмотреть функцию $g:G_{k,d}\times\mathbb R^{d-k}\longrightarrow\mathbb R$, тогда
		$$g\in L_2(G_{k,d}\times \mathbb R^{d-k})\Rightarrow g(\pi,\cdot)\in L_2(\mathbb R^{d-k})$$ для почти всех $\pi\in G_{k,d}$. Это утверждение Теоремы Фубини.
    \item $g(\pi,\cdot)\in L_2(\mathbb R^{d-k})$ для п.в. $\pi\in G_{k,d}$ $\Rightarrow \hat {g_\pi}\in L_2(\mathbb R^{d-k})$ для п.в. $\pi\in G_{k,d}$ (где $g_\pi(x)=g(\pi,x)$).
\end{itemize}

Однако из того, что $\hat {g_\pi}\in L_2(\pi^\perp)$ для п.в. $\pi\in G_{k,d}$, вообще говоря, не следует, что $\hat {g_\pi}\in L_2(TG_{k,d})$


\end{document}
